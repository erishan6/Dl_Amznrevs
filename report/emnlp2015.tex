%
% File emnlp2015.tex
%
% Contact: daniele.pighin@gmail.com
%%
%% Based on the style files for ACL-2015, which were, in turn,
%% Based on the style files for ACL-2014, which were, in turn,
%% Based on the style files for ACL-2013, which were, in turn,
%% Based on the style files for ACL-2012, which were, in turn,
%% based on the style files for ACL-2011, which were, in turn,
%% based on the style files for ACL-2010, which were, in turn,
%% based on the style files for ACL-IJCNLP-2009, which were, in turn,
%% based on the style files for EACL-2009 and IJCNLP-2008...

%% Based on the style files for EACL 2006 by
%%e.agirre@ehu.es or Sergi.Balari@uab.es
%% and that of ACL 08 by Joakim Nivre and Noah Smith

\documentclass[11pt,a4paper]{article}
\usepackage{acl2015}
\usepackage{times}
\usepackage{url}
\usepackage{latexsym}

%\setlength\titlebox{5cm}

% You can expand the titlebox if you need extra space
% to show all the authors. Please do not make the titlebox
% smaller than 5cm (the original size); we will check this
% in the camera-ready version and ask you to change it back.


\title{Sentiment Analysis of Amazon Product Reviews Using Domain Adversarial Training}

\author{First Author \\
  Affiliation / Address line 1 \\
  Affiliation / Address line 2 \\
  Affiliation / Address line 3 \\
  {\tt email@domain} \\\And
  Second Author \\
  Affiliation / Address line 1 \\
  Affiliation / Address line 2 \\
  Affiliation / Address line 3 \\
  {\tt email@domain} \\}

\date{}

\begin{document}
\maketitle
\begin{abstract}
  In this project, sentiment analysis of Amazon product reviews was
  done across three different domains(viz. books, music and dvd). We
  have used domain adversarial training for the neural network.
\end{abstract}

\section{Introduction}

intro, basic motivation

\section{Classification with CNN}

Describe the baseline model

\section{Domain Adversial Training}

Describe the concept.

\subsection{Method}
Describe the specific network.

\subsection{Research Question}

\section{Dataset}
The data is from 3 different domains (books, dvds, movies) and 3 different languages (German, English, French).
We have only used the English dataset for this work. The data has been obtained from: \url{https://www.uni-weimar.de/en/media/chairs/computer-science-and-media/webis/corpora/corpus-webis-cls-10/} and it is described in~\cite{PB:2010}.

For each language and each domain there are 2000 training examples and 2000 test examples. The ratings to positive/negative sentiment are mapped as follows: 1.0, 2.0: negative, 4.0, 5.0: positive. There are no instances with a rating of 3.0. The datasets are balanced, this means for each language and each domain 50\% of the items have a positive and 50\% have a negative rating.

\section{Experiments and Results}



\section{Conclusion}



% include your own bib file like this:
%\bibliographystyle{acl}
%\bibliography{acl2015}

\begin{thebibliography}{}
% this paper describes the dataset.
\bibitem[\protect\citename{{Prettenhofer and Stein}}2010]{PB:2010}
Prettenhofer, Peter and Stein, Benno
\newblock 2010.
\newblock {\em Cross-language Text Classification Using Structural Correspondence Learning}.
\newblock Proceedings of the 48th Annual Meeting of the Association for Computational Linguistics.

\bibitem[\protect\citename{{Chen \bgroup et al.\egroup }}2016]{Chen:2016}
Xilun Chen and Ben Athiwaratkun and Yu Sun and Kilian Q. Weinberger and Claire Cardie.
\newblock 2016.
\newblock {\em Adversarial Deep Averaging Networks for Cross-Lingual Sentiment Classification}.
\newblock CoRR.

\bibitem[\protect\citename{{Ganin \bgroup et al.\egroup }}2016]{Ganin:2016}
Ganin, Yaroslav and Ustinova, Evgeniya and Ajakan, Hana and Germain, Pascal and Larochelle, Hugo and Laviolette, Fran\c{c}ois and Marchand, Mario and Lempitsky, Victor.
\newblock 2016.
\newblock {\em Domain-adversarial Training of Neural Networks},
  17(1):2096--2030.
\newblock J. Mach. Learn. Res.

\bibitem[\protect\citename{Britz}]{Britz}
Denny Britz.
\newblock 2015.
\newblock {\em Implementing a CNN for Text Classification in TensorFlow}.
\newblock http://www.wildml.com/2015/12/implementing-a-cnn-for-text-classification-in-tensorflow/.

\bibitem[\protect\citename{Britz Github}]{Britz:github}
Denny Britz.
\newblock 2015.
\newblock {\em dennybritz/cnn-text-classification-tf}.
\newblock https://github.com/dennybritz/cnn-text-classification-tf.


\end{thebibliography}

\end{document}
